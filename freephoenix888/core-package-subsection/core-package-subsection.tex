\subsection{@deep-foundation/core}
\subsubsection{Описание}\hypertarget{Core.Description}{}
@deep-foundation/core это встроенный пакет, которые
содержит самые важные типы для работы с Deep
\subsubsection{Связи}
\paragraph*{Type}\hypertarget{Core.Type.Description}{}
Тип Type используется для создания новых типов
\paragraph*{Contain}\hypertarget{Core.Contain.Description}{}
Тип Contain используется для обозначения
принадлежности, а так же для присваивания имени связи. Например пакет, что бы
содержать в себе другие связи - имеет
\hyperlink{Core.Contain.Description}{Contain} от себя
до другой связи. Если вы хотите дать имя связи - обновите значение экземпляра
\paragraph*{Package}\hypertarget{Core.Package.Description}{}
Тип Package олицетворяет пакет, который содержит в себе связи и/или зависит от
связей из других пакетов
\hyperlink{Core.Contain.Description}{Contain}
\paragraph*{SyncTextFile}\hypertarget{Core.SyncTextFile.Description}{}
Тип SyncTextFile используется для хранения
любого строкового значения
\paragraph*{Any}\hypertarget{Core.Any.Description}{}
Тип Any может использоваться для ограничений
\subparagraph*{Начало/конец типа}
Например есть при создании типа А вы можете указать, что его началом
обязательно должен быть Б, а концом может быть любая связь, для этого концом у
типа будет указан Any
\subparagraph*{Обрабатываемый тип}
Any может использоваться как обрабатываемый тип, в таком случае обработчик
будет обрабатывать экземпляры всех типов
\paragraph*{Handler}\hypertarget{Core.Handler.Description}{}
Тип Handler используется для создания обработчика.
Его началом является экземпляр \hyperlink{supports.Description}{Supports}, а
концом
любая связь, содержащая код в строковом значении
\paragraph*{Supports}\hypertarget{Core.Supports.Description}{}
Тип Supports используется для создания среды
выполнения обработчика. Примером среды выполнения обработчика является
\hyperlink{Core.dockerSupportsJs.Description}{dockerSupportsJs}
\paragraph*{dockerSupportsJs}\hypertarget{Core.dockerSupportsJs.Description}{}
Связь dockerSupportsJs используется для
поддержки JavaScript в Docker. Что бы использовать эту среду для обработчика,
нужно сделать эту связь началом связи
\hyperlink{Core.Handler.Description}{Handler}
\paragraph*{HandleInsert}\hypertarget{Core.HandleInsert.Description}{}
Тип HandleInsert используется для обработки
вставки экземпляров определённого типа. Например, что бы обрабатывать вставку
экземпляров типа \hyperlink{Core.Type.Description}{Type}, нужно создать
HandleInsert с
началом типом \hyperlink{Core.Type.Description}{Type} и концом обработчиком, то
есть
экземпляром \hyperlink{Core.Handler.Description}{Handler}
\paragraph*{HandleUpdate}\hypertarget{Core.HandleUpdate.Description}{}
Тип HandleUpdate используется для обработки
обновлений экземпляров определённого типа. Например, что бы обрабатывать
обновления экземпляров типа \hyperlink{Core.Type.Description}{Type}, нужно
создать
HandleUpdate с началом \hyperlink{Core.Type.Description}{Type} и концом
обработчиком,
то есть экземпляром \hyperlink{Core.Handler.Description}{Handler}
\paragraph*{HandleUpdate}\hypertarget{Core.HandleDelete.Description}{}
Тип HandleDelete используется для обработки
удалений экземпляров определённого типа. Например, что бы обрабатывать удаления
экземпляров типа \hyperlink{Core.Type.Description}{Type}, нужно создать
HandleDelete с
началом типом \hyperlink{Core.Type.Description}{Type} и концом обработчиком, то
есть
экземпляром \hyperlink{Core.Handler.Description}{Handler}
\paragraph*{Tree}\hypertarget{Core.Tree.Description}{}
Тип Tree используется для создания дерева
\paragraph*{TreeIncludeDown}\hypertarget{Core.TreeIncludeDown.Description}{}
Тип TreeIncludeDown используется для
включения в дерево связи, по направлению вниз
\paragraph*{TreeIncludeUp}\hypertarget{Core.TreeIncludeUp.Description}{}
Тип TreeIncludeUp используется для включения в
дерево связи, по направлению вверх
\paragraph*{TreeIncludeNode}\hypertarget{Core.TreeIncludeNode.Description}{}
Тип TreeIncludeNode используется для
включения в дерево связи, у которой нет начала и конца
\paragraph*{Route}\hypertarget{Core.Route.Description}{}
Тип Route используется для создания веб-маршрута
\paragraph*{HandleRoute}\hypertarget{Core.HandleRoute.Description}{}
Связь HandleRoute используется для обработки
веб-маршрута
\paragraph*{JSDockerIsolationProvider}\hypertarget{Core.JSDockerIsolationProvider.Description}{}
Связь JSDockerIsolationProvider содержит название docker container'a,
используемого для запуска JavaScript кода
\paragraph*{Schedule}\hypertarget{Core.Schedule.Description}{}
Тип Schedule используется для создания расписания
\paragraph*{HandleSchedule}\hypertarget{Core.HandleSchedule.Description}{}
Связь HandleSchedule используется для обработки расписания
\paragraph*{Router}\hypertarget{Core.Router.Description}{}
Тип Router используется для создания роутера
\paragraph*{RouterStringUse}\hypertarget{Core.RouterStringUse.Description}{}
Связь RouterStringUse используется для
указания веб-маршрута, который должен обрабатываться
\paragraph*{Port}\hypertarget{Core.Port.Description}{}
Тип Port используется для создания порта
\paragraph*{RouterListening}\hypertarget{Core.RouterListening.Description}{}
Связь RouterListening используется для
указания порта, который должен обрабатываться
\paragraph*{Space}\hypertarget{Core.Space.Description}{}
Тип Space используется для создания пространства
\paragraph*{Value}
Тип Value используется для обозначения типа значения у связи
Например, что бы указать, что экземпляры Message могут иметь строковое значение - нужно Value от Message до \hyperlink{Core.String.Description}{String}
\paragraph*{String}
Тип String используется для обозначения строкового значения. Например, что бы указать, что экземпляры Message могут иметь строковое значение - нужно Value от Message до String
\paragraph*{Number}
Тип String используется для обозначения числового значения. Например, что бы указать, что экземпляры Message могут иметь числовое значение - нужно Value от Message до Number
\paragraph*{Object}
Тип String используется для обозначения строкового значения. Например, что бы указать, что экземпляры Message могут иметь объектное значение - нужно Value от Message до Object