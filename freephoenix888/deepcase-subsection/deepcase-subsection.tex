\subsection{DeepCase}
\subsubsection{Описание}
DeepCase\hypertarget{DeepCase.Description}{} является графическим визуальным
интерфейсом Deep.
\hyperlink{DeepCase.Description}{DeepCase} позволяет взаимодействовать со
связями:
вставлять,
обновлять, удалять, просматривать, делать выборку связей,
путешествовать по ним, используя traveller, и ещё множество возможностей
\hyperlink{DeepCase.Description}{DeepCase} это npm пакет, который позволяет
встраивать
его компоненты, включая
отдельные клиентские обработчики, используя ClientHandler компонент в любой
проект, использующий npm пакет React
\subsubsection{Основные возможности}
\paragraph{Виуализация связи}
Визуализация конкретных связей реализована при помощи клиентских обработчиков.
Что бы увидеть визуализацию связи, нужно нажать на неё
\paragraph{Вставить связь}
Для того, что бы \hypertarget{DeepCase.InsertLink.Description}{вставить связь},
нужно
нажать правкую кнопку мыши в пустом
месте и в открытом контекстном меню нажать на "Insert". После этого откроется
меню вставки связи, в котором есть поле для ввода что бы фильтровать связи по
названию, а так же сами связи ниже поля для ввода. Введите название нужной
связи, например \hyperlink{Core.Type.Description}{Type} и ниже найдите связь с
таким
названием. Вы можете
заметить, что там есть как минимум два \hyperlink{Core.Type.Description}{Type},
один со
значком звёздочки, а
другой со значком папки. Связь с значком папки это
\hyperlink{Core.Contain.Description}{Contain}, следовательно нажав
на стрелку влево около этой связи, мы сможем увидеть откуда ведёт
\hyperlink{Core.Contain.Description}{Contain}
(после нажатия активная связь появляется над полем для ввода), а нажав на
стрелку вправо сможем увидеть на какую связь указывает
\hyperlink{Core.Contain.Description}{Contain}, следовательно
\hyperlink{Core.Contain.Description}{Contain} будет указывать на ту самую связь
со
звёздочкой, то есть вы можете
выбрать \hyperlink{Core.Type.Description}{Type} либо нажав на стрелку вправо у
его
\hyperlink{Core.Contain.Description}{Contain}'a, либо нажав на сам
\hyperlink{Core.Type.Description}{Type} (на название). Удостоверьтесь, что над
полем для
ввода отображается нужная
связь и после этого нажмите на галочку справа внизу от меню вставки, что бы
активировать вставку этой связи. После этого вы увидите уведомление слева внизу
о том, что вы сейчас вставляете связь определённого типа, а так же сказано
какая это связь, Node (точка), или же она является связью между двумя
определёнными типами связей. Если это Node вам предложат нажать в любом
свободном месте, что бы вставить связь, если же это не Node, но нажать не
отпуская на связь и привести появившуюся линию к другой связи, что бы создать
связь между этими связями. Так же можно нажать крестик на уведомлении что бы
отменить вставку
\paragraph{Обновить связь}
Для того, что бы обновить связь, нужно открыть контекстное меню связи и нажать
на кнопку Update, после чего протянуть линию от нового начала до нового конца
связи
Заметка: если не протягивать линию, а нажать на связь один раз, то она станет и
началом и концом связи
\paragraph{Удалить связь}
Для того, что бы удалить связь, нужно открыть контекстное меню связи и нажать
на кнопку "Delete"
\paragraph{Удаление ниже по дереву
  containTree}\hypertarget{containTree.Deletion.Example}{}
Для того, что бы удалить все связи ниже по дереву
\hyperlink{containTree.Description}{containTree}, нужно открыть контекстное
меню связи
и нажать на кнопку "Delete Down"
\paragraph*{Traveller} 
\hyperlink{Traveller.Description}{Traveller} используется для путешествия по
связям.
При нажатии на кнопку Traveller в контекстном меню вам сначала нужно выбрать в
каком направлении вы хотите путешествовать, а затем выставить ограничение
путешествия.
Если вы путешествуете вверх или вниз по дереву, следовательно после этого вам
предоставят возможность выбрать дерево, по которому вы будете путешествовать.
Если вы путешествуете по чему-либо другому, например in/out, то после этого вам
предоставят возможность выбрать отфильтровать эти связи по определённому
признаку, либо же показать все связи
\paragraph*{Редактор}
\hypertarget{Editor.Description}{Редактор} используется для:
\begin{itemize}
  \item Редактирования значения связей
  \item Управлением обработчиками
  \item Просмотра результатов обработчиков
\end{itemize}
