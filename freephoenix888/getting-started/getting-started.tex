\subsection{Начало работы}
\subsubsection{Запуск Deep}
\begin{itemize}
  \item Перейдите в \hyperref{https://github.com/deep-foundation/dev?tab=readme-ov-file#readme}{readme монорепозитория dev}
  \item Нажмите на кнопку \hyperref{https://gitpod.io/#https://github.com/deep-foundation/dev}{Gitpod} для создания нового рабочего пространства Gitpod с установленным монорепозиторием dev. Обратите внимание, что Gitpod может попросить вас зарегистрироваться или войти.
\end{itemize}

\subsubsection{Открытие DeepCase}
\begin{itemize}
  \item Когда рабочее пространство Gitpod с монорепозиторием dev будет инициализировано, нажмите на кнопку "Ports" в правом нижнем углу экрана и откройте порт 3007, кликнув по его URL или на кнопку "Open Browser" (имеет иконку глобуса).
  \item Теперь у вас открыт DeepCase. Это визуальный интерфейс для Deep. Вы можете использовать его для создания, обновления, удаления, просмотра, выбора, перемещения по ссылкам и многого другого.
\end{itemize}
\href{https://youtu.be/0lbcPGV17mQ}{Видео-гайд}
