\subsection{Начало работы}
\subsubsection{Запуск Deep}
\begin{itemize}
  \item Перейдите в \hyperref{https://github.com/deep-foundation/dev?tab=readme-ov-file#readme}{readme монорепозитория dev}
  \item Нажмите на кнопку \hyperref{https://gitpod.io/#https://github.com/deep-foundation/dev}{Gitpod} для создания нового рабочего пространства Gitpod с установленным монорепозиторием dev. Обратите внимание, что Gitpod может попросить вас зарегистрироваться или войти.
\end{itemize}

\subsubsection{Открытие DeepCase}
\begin{itemize}
  \item Когда рабочее пространство Gitpod с монорепозиторием dev будет инициализировано, нажмите на кнопку "Ports" в правом нижнем углу экрана и откройте порт 3007, кликнув по его URL или на кнопку "Open Browser" (имеет иконку глобуса).
  \item Теперь у вас открыт DeepCase. Это визуальный интерфейс для Deep. Вы можете использовать его для создания, обновления, удаления, просмотра, выбора, перемещения по ссылкам и многого другого.
\end{itemize}
\subsubsection{Авторизация как администратор}
\begin{itemize}
  \item Пропутешействуйте до типов вашего пользователя: Откройте контекстное меню вашего пользователя, нажмите на кнопку traveller, затем 'types'. Вы увидите типы вашего пользователя, то есть \hyperlink{Core.User.Description}{User} и его тип \hyperlink{Core.Type.Description}{Type}
  \item Пропутешействуйте до связей типизированных типом \hyperlink{Core.User.Description}{User}: Откройте контекстное меню типа \hyperlink{Core.User.Description}{User} и нажмите на кнопку 'typed'. вы увидите все связи, которые имеют этот тип, в том числе там будет пользователь с именем admin
  \item Авторизуйтесь как admin: Откройте контекстное меню связи admin и нажмите Login
\end{itemize}
\href{https://youtu.be/6UgOiIZMYr8}{Видео-гайд}
\subsubsection{Создание пользователя, выдача ему прав, авторизация}
\begin{itemize}
  \item \hyperlink{DeepCase.InsertLink.Description}{Вставьте} связь типа \hyperlink{Core.User.Description}{User} и \hyperlink{FAQ.HowToSetName}{назовите} её myUser
  \item \hyperlink{DeepCase.InsertLink.Description}{Вставьте} связь типа \hyperlink{Core.Join.Description}{Join} от myUser до admin
  \item Авторизуйтесь как myUser
\end{itemize}
\href{https://youtu.be/TeGln__84y8}{Видео-гайд}

