\subsection{Обработчики}
% TODO: Add examples with editor->handlers
\subsubsection{Виды обработчиков}
\begin{itemize}
  \item \textbf{Серверные обработчики баз данных и событий над связями}
  \item \textbf{Серверные обработчики веб маршрута и порта}
  \item \textbf{Серверные обработчики порта}
  \item \textbf{Серверные обработчики по расписанию}
  \item \textbf{Клиентские обработчики}
\end{itemize}

\subsubsection{Серверные обработчики баз данных и событий над связями}
\paragraph{Обрабатываемые события:}
\begin{itemize}
  \item Вставка
  \item Обновление
  \item Удаление
\end{itemize}
\paragraph{Виды обработчиков:}
\begin{itemize}
  \item \textbf{Синхронные}
  \item \textbf{Асинхронные}
\end{itemize}

\paragraph{Синхронные обработчики}
\subparagraph{Описание}

Синхронные серверные обработчики выполняются внутри транзации внутри базы
данных и в случае ошибки
транзакция отменяется
\subparagraph{Примеры}
\paragraph{Создание обработчика с помощью редактора}
\begin{itemize}
      \item \hyperlink{DeepCase.InsertLink.Description}{Вставьте} связь типа
            \hyperlink{Core.Type.Description}{Type} и
            \hyperlink{FAQ.HowToSetName}{назовите} её Email
      \item \hyperlink{DeepCase.InsertLink.Description}{Вставьте} связь типа
            \hyperlink{Core.SyncTextFile.Description}{SyncTextFile} с следующим
            значением:
            \begin{lstlisting}
    ({ data: { newLink: emailLink } }) => {
      const email = emailLink.value?.value;
      if (!email) {
      throw new Error(`Link ${emailLink.id} has not value`);
      }
      if (!email.includes("@")) {
      throw new Error(`Email ${email} does not include @`);
      }
    };
                \end{lstlisting}
            и \hyperlink{FAQ.HowToSetName}{назовите} её myHandlerCode
      \item \hyperlink{DeepCase.InsertLink.Description}{Вставьте} связь типа
            \hyperlink{Core.Handler.Description}{Handler}
            \hyperlink{FAQ.HowToInsertLinkWithFromAndTo}{с началом}
            \hyperlink{Core.plv8SupportsJs.Description}{dockerSupportsJs}
            \hyperlink{FAQ.HowToInsertLinkWithFromAndTo}{и
              концом}
            myHandlerCode \hyperlink{FAQ.HowToSetName}{назовите} её myHandler
      \item \hyperlink{DeepCase.InsertLink.Description}{Вставьте} связь типа
            Email и \hyperlink{FAQ.HowToSetName}{назовите} её email
      \item \hyperlink{DeepCase.UpdateLink.Description}{Обновите} значение
            связи email на значение "test"
      \item Созерцайте ошибку, говорящую о том, что Email не содержит @
      \item \hyperlink{DeepCase.UpdateLink.Description}{Обновите} значение
            связи email на значение "test@deep"
      \item Созерцайте отсутствие ошибки
    \end{itemize}
\paragraph{Создание обработчика с помощью связей}
\begin{itemize}
  \item \hyperlink{DeepCase.InsertLink.Description}{Вставьте} связь типа
        \hyperlink{Core.Type.Description}{Type} и
        \hyperlink{FAQ.HowToSetName}{назовите} её Email
  \item \hyperlink{DeepCase.InsertLink.Description}{Вставьте} связь типа
        \hyperlink{Core.SyncTextFile.Description}{SyncTextFile} с следующим
        значением:
        \begin{lstlisting}
({ data: { newLink: emailLink } }) => {
  const email = emailLink.value?.value;
  if (!email) {
      throw new Error(`Link ${emailLink.id} has not value`);
  }
  if (!email.includes("@")) {
      throw new Error(`Email ${email} does not include @`);
  }
};
            \end{lstlisting}
        и \hyperlink{FAQ.HowToSetName}{назовите} её myHandlerCode
  \item \hyperlink{DeepCase.InsertLink.Description}{Вставьте} связь типа
        \hyperlink{Core.Handler.Description}{Handler}
        \hyperlink{FAQ.HowToInsertLinkWithFromAndTo}{с началом}
        \hyperlink{Core.plv8SupportsJs.Description}{dockerSupportsJs}
        \hyperlink{FAQ.HowToInsertLinkWithFromAndTo}{и
          концом}
        myHandlerCode \hyperlink{FAQ.HowToSetName}{назовите} её myHandler
  \item \hyperlink{DeepCase.InsertLink.Description}{Вставьте} связь типа
        Email и \hyperlink{FAQ.HowToSetName}{назовите} её email
  \item \hyperlink{DeepCase.UpdateLink.Description}{Обновите} значение
        связи email на значение "test"
  \item Созерцайте ошибку, говорящую о том, что Email не содержит @
  \item \hyperlink{DeepCase.UpdateLink.Description}{Обновите} значение
        связи email на значение "test@deep"
  \item Созерцайте отсутствие ошибки
\end{itemize}

\paragraph{Асинхронные обработчики}
\subparagraph{Описание}
Асинхронные серверные обработчики выполняются после транзакции, в результате
приводя к появлению следующих связей: \\
\hypertarget{Handlers.Async.Result}{}Связь-триггер --Then-> Promise
--Resolved/Rejected-> PromiseResult
Обратите внимание на Resolved/Rejected. В случае успешного выполнения
используется Resolved, иначе Rejected

\hypertarget{Handlers.Async.HowToGetResult}{}Результат выполнения асинхронного
обработчика находится ниже по дереву
promiseTree от связи-триггера. В визуальном интерфейсе
\hyperlink{DeepCase.Description}{DeepCase} можно увидеть
результат используя контекстное меню->traveler->down->promiseTree

Асинхронные обработчики могут запускаться в разных средах. Одна из популярных -
докер среда, которая использует определенный инструмент, например - node для
запуска JavaScript кода.
\subparagraph{Примеры:}
\paragraph{Создание обработчика с помощью редактора}
Заметка: создание обработчика с помощью редактора тоже порождает связи, как в примере с созданием обработчика используя связи
\begin{itemize}
      \item \hyperlink{DeepCase.InsertLink.Description}{Вставьте} связь типа
            \hyperlink{Core.Type.Description}{Type} и
            \hyperlink{FAQ.HowToSetName}{назовите} её Email
      \item \hyperlink{DeepCase.InsertLink.Description}{Вставьте} связь типа
            \hyperlink{Core.SyncTextFile.Description}{SyncTextFile} с следующим
            значением:
            \begin{lstlisting}
async ({ data: { newLink: emailLink } }) => {
  const email = emailLink.value?.value;
  if (!email) {
    throw new Error(`Link ${emailLink.id} has not value`);
  }
  if (!email.includes("@")) {
    throw new Error(`Email ${email} does not include @`);
  }
};
                \end{lstlisting}
            и \hyperlink{FAQ.HowToSetName}{назовите} её myHandlerCode
      \item \hyperlink{DeepCase.InsertLink.Description}{Вставьте} связь типа
            \hyperlink{Core.Handler.Description}{Handler}
            \hyperlink{FAQ.HowToInsertLinkWithFromAndTo}{с началом}
            \hyperlink{Core.dockerSupportsJs.Description}{dockerSupportsJs}
            \hyperlink{FAQ.HowToInsertLinkWithFromAndTo}{и
              концом}
            myHandlerCode \hyperlink{FAQ.HowToSetName}{назовите} её myHandler
      \item \hyperlink{DeepCase.InsertLink.Description}{Вставьте} связь типа
            Email и \hyperlink{FAQ.HowToSetName}{назовите} её email
      \item \hyperlink{DeepCase.UpdateLink.Description}{Обновите} значение
            связи email на значение "test"
      \item Созерцайте ошибку, говорящую о том, что Email не содержит @
      \item \hyperlink{DeepCase.UpdateLink.Description}{Обновите} значение
            связи email на значение "test@deep"
      \item Созерцайте отсутствие ошибки
    \end{itemize}
\paragraph{Создание обработчика с помощью связей}
\begin{itemize}
  \item \hyperlink{DeepCase.InsertLink.Description}{Вставьте} связь типа
        \hyperlink{Core.Type.Description}{Type} и
        \hyperlink{FAQ.HowToSetName}{назовите} её Email
  \item \hyperlink{DeepCase.InsertLink.Description}{Вставьте} связь типа
        \hyperlink{Core.SyncTextFile.Description}{SyncTextFile} с следующим
        значением:
        \begin{lstlisting}
async ({ data: { newLink: emailLink } }) => {
  const email = emailLink.value?.value;
  if (!email) {
    throw new Error(`Link ${emailLink.id} has not value`);
  }
  if (!email.includes("@")) {
    throw new Error(`Email ${email} does not include @`);
  }
};
            \end{lstlisting}
        и \hyperlink{FAQ.HowToSetName}{назовите} её myHandlerCode
  \item \hyperlink{DeepCase.InsertLink.Description}{Вставьте} связь типа
        \hyperlink{Core.Handler.Description}{Handler}
        \hyperlink{FAQ.HowToInsertLinkWithFromAndTo}{с началом}
        \hyperlink{Core.dockerSupportsJs.Description}{dockerSupportsJs}
        \hyperlink{FAQ.HowToInsertLinkWithFromAndTo}{и
          концом}
        myHandlerCode \hyperlink{FAQ.HowToSetName}{назовите} её myHandler
  \item \hyperlink{DeepCase.InsertLink.Description}{Вставьте} связь типа
        Email и \hyperlink{FAQ.HowToSetName}{назовите} её email
  \item \hyperlink{DeepCase.UpdateLink.Description}{Обновите} значение
        связи email на значение "test"
  \item \hyperlink{Handlers.Async.HowToGetResult}{Просмотрите результат
          асинхронного хендлера для связи email}
  \item Видим \hyperlink{Handlers.Async.Result}{результат асинхронного
          обработчика в виде связей}
  \item \hyperlink{DeepCase.UpdateLink.Description}{Обновите} значение
        связи email на значение "test@deep"
  \item \hyperlink{Handlers.Async.HowToGetResult}{Просмотрите результат
          асинхронного хендлера для связи email}
  \item Видим \hyperlink{Handlers.Async.Result}{результат асинхронного
          обработчика в виде связей}
\end{itemize}

\paragraph{Справочник}
\subparagraph{Параметры обработчика}
\begin{enumerate}
  \item Объект, содержащий поля:
        \subitem{deep} - Дип Клиент
        \subitem{require} - функция для импортирования npm пакетов
        \subitem{data} - объект, содержащий поля:
        \subsubitem{newLink} - связь-триггер, на событие которой сработал
        обработчик
        \subsubitem{oldLink} - связь до операции
        \subsubitem{triggeredByLinkId} - связь, которая произвела операцию.
        Например связь пользователя
\end{enumerate}
\subsubsection{Серверные обработчики веб маршрута и порта}
\paragraph{Описание}
Серверные обработчики веб-маршрута и порта предоставляют возможность
эффективной обработки запросов по определенным веб-маршрутам и портам. Для
иллюстрации, представьте, что адрес нашего веб-сервера - deep-foundation.com.
Теперь мы можем легко настроить обработку определенного пути, например,
my-path, на конкретном порту, например, 5000, следуя данной конструкции:
deep-foundation.com/my-path:5000.

Эта гибкость позволяет точно настраивать поведение сервера в зависимости от
веб-маршрута и порта, что делает наш проект мощным инструментом для управления
трафиком и обработки запросов с высокой точностью.

Серверные обработчики реализованы с помощью npm пакета express
\paragraph{Примеры}
\subparagraph{Обработчик, возвращающий "ok" в любом случае}
\begin{itemize}
  \item \hyperlink{DeepCase.InsertLink.Description}{Вставьте} связь типа
        \hyperlink{Core.SyncTextFile.Description}{SyncTextFile} с таким
        кодом:
        \begin{lstlisting}
(request, response) => {
  response.send("ok");
}
\end{lstlisting}
  \item \hyperlink{DeepCase.InsertLink.Description}{Вставьте} связь типа
        \hyperlink{Core.Handler.Description}{Handler} от
        \hyperlink{Core.dockerSupportsJs.Description}{dockerSupportsJs}
        до нашего \hyperlink{Core.SyncTextFile.Description}{SyncTextFile}
  \item \hyperlink{DeepCase.InsertLink.Description}{Вставьте} связь типа
        \hyperlink{Core.Route.Description}{Route}
  \item \hyperlink{DeepCase.InsertLink.Description}{Вставьте} связь типа
        \hyperlink{Core.HandleRoute.Description}{HandleRoute} от нашего
        \hyperlink{Core.Route.Description}{Route} до нашего
        \hyperlink{Core.Handler.Description}{Handler}
  \item \hyperlink{DeepCase.InsertLink.Description}{Вставьте} связь типа
        \hyperlink{Core.Router.Description}{Router}
  \item \hyperlink{DeepCase.InsertLink.Description}{Вставьте} связь типа
        \hyperlink{Core.RouterStringUse.Description}{RouterStringUse} от
        нашего
        \hyperlink{Core.Route.Description}{Route} до нашего
        \hyperlink{Core.Router.Description}{Router},
        с
        значением
        веб-маршрута, который должен обрабатываться, например /my-path
  \item \hyperlink{DeepCase.InsertLink.Description}{Вставьте} связь типа
        \hyperlink{Core.Port.Description}{Port} с значением порта, который
        должен
        обрабатываться, например
        5000
  \item \hyperlink{DeepCase.InsertLink.Description}{Вставьте} связь типа
        \hyperlink{Core.RouterListening.Description}{RouterListening} от
        нашего
        \hyperlink{Core.Router.Description}{Router} до нашего
        \hyperlink{Core.Port.Description}{Port}
\end{itemize}

\subparagraph{Обработчик, обрабатывающий входящий запрос}
В данном примере обработчик будет проверять поле isOk в входящем запрос, если
он true, то мы ответим строкой "ok", иначе "not ok"
\begin{itemize}
  \item \hyperlink{DeepCase.InsertLink.Description}{Вставьте} связь типа
        \hyperlink{Core.SyncTextFile.Description}{SyncTextFile} с таким
        кодом:
        \begin{lstlisting}
async (
  request,
  response,
) => {
  if(request.isOk) {
    response.send("ok")
  } else {
    response.send("not ok")
  }
}
\end{lstlisting}
  \item \hyperlink{DeepCase.InsertLink.Description}{Вставьте} связь типа
        \hyperlink{Core.Handler.Description}{Handler} от
        \hyperlink{Core.dockerSupportsJs.Description}{dockerSupportsJs}
        до нашего \hyperlink{Core.SyncTextFile.Description}{SyncTextFile}
  \item \hyperlink{DeepCase.InsertLink.Description}{Вставьте} связь типа
        \hyperlink{Core.Route.Description}{Route}
  \item \hyperlink{DeepCase.InsertLink.Description}{Вставьте} связь типа
        \hyperlink{Core.HandleRoute.Description}{HandleRoute} от нашего
        \hyperlink{Core.Route.Description}{Route} до нашего
        \hyperlink{Core.Handler.Description}{Handler}
  \item \hyperlink{DeepCase.InsertLink.Description}{Вставьте} связь типа
        \hyperlink{Core.Router.Description}{Router}
  \item \hyperlink{DeepCase.InsertLink.Description}{Вставьте} связь типа
        \hyperlink{Core.RouterStringUse.Description}{RouterStringUse} от
        нашего
        \hyperlink{Core.Route.Description}{Route} до нашего
        \hyperlink{Core.Router.Description}{Router},
        с
        значением
        веб-маршрута, который должен обрабатываться, например /my-path
  \item \hyperlink{DeepCase.InsertLink.Description}{Вставьте} связь типа
        \hyperlink{Core.Port.Description}{Port} с значением порта, который
        должен
        обрабатываться, например
        5000
  \item \hyperlink{DeepCase.InsertLink.Description}{Вставьте} связь типа
        \hyperlink{Core.RouterListening.Description}{RouterListening} от
        нашего
        \hyperlink{Core.Router.Description}{Router} до нашего
        \hyperlink{Core.Port.Description}{Port}
\end{itemize}

\paragraph{Параметры обработчика}

\urldef{\requrl}\url{https://expressjs.com/en/api.html#req}
\urldef{\resurl}\url{https://expressjs.com/en/api.html#res}
\urldef{\midwareurl}\url{https://expressjs.com/en/guide/using-middleware.html}

\begin{enumerate}
  \item{request} \texttt{Объект, который представляет HTTP-запрос и имеет
    свойства для
    строки запроса, параметров, тела, HTTP-заголовков и т. д. Подробное
    описание:
    \requrl}
  \item{response} \texttt{Объект, который представляет HTTP-ответ, который
    приложение
    Express отправляет при получении HTTP-запроса. Подробное описание:
    \resurl}
  \item{otherArgs} Объект, содержащий поля:
  \begin{enumerate}
    \item \texttt{deep} Дип Клиент
    \item \texttt{require} - функция для импортирования npm пакетов
    \item \texttt{next} - функция для передачи работы другому
          обработчику. Подробное описание: \midwareurl
  \end{enumerate}
\end{enumerate}
\subsubsection{Серверные обработчики порта}
\paragraph{Описание}
Серверные обработчики порта предоставляют возможность обрабатывать запросы на
определенном порту используя заданный docker container. Для иллюстрации,
представьте, что адрес нашего веб-сервера - deep-foundation.com. Теперь мы
можем легко настроить обработку определенного порта, например, 5000, следуя
данной конструкции: deep-foundation.com:5000. Эта гибкость позволяет точно
настраивать поведение сервера в зависимости от порта, что делает наш проект
мощным инструментом для управления трафиком и обработки запросов с высокой
точностью. Серверные обработчики реализованы с помощью npm пакета express
\paragraph{Примеры}
\subparagraph{Пример обработчика, использующего Docker Container под названием
  JS Docker Isolation Container}
Docker Container под названием JS Docker Isolation Container обрабатывает
маршрут /healtz. В нашем случае мы будем обрабатывать порт 5000, следовательно
наш маршрут будет выглядеть так: deep-foundation.com:5000/healtz (вместо
deep-foundation.com используйте свой адрес, на котором находится deep)
\begin{itemize}
  \item \hyperlink{DeepCase.InsertLink.Description}{Вставьте} связь типа
        \hyperlink{Core.SyncTextFile.Description}{SyncTextFile} с таким
        кодом:
        \begin{lstlisting}
(request, response) => {
  response.send("ok");
}
\end{lstlisting}
  \item \hyperlink{DeepCase.InsertLink.Description}{Вставьте} связь типа
        \hyperlink{Core.Port.Description}{Port} и
        \hyperlink{FAQ.HowToSetName}{назовите} её myPort
  \item \hyperlink{DeepCase.UpdateLink.Description}{Обновите} значение
        связи
        myPort на значение 5000
  \item \hyperlink{DeepCase.InsertLink.Description}{Вставьте} связь типа
        \hyperlink{Core.HandlePort.Description}{HandlePort}
        \hyperlink{FAQ.HowToInsertLinkWithFromAndTo}{с началом}
        myPort \hyperlink{FAQ.HowToInsertLinkWithFromAndTo}{и концом}

        \hyperlink{Core.JSDockerIsolationProvider.Description}{JSDockerIsolationProvider}
  \item Перейдите по адресу deep-foundation.com:5000/healtz (вместо
        deep-foundation.com используйте свой адрес, на котором находится
        deep) и
        увидите ответ "ok" % TODO: Точно ок?
\end{itemize}

\subsubsection*{Серверные обработчики по расписанию}
Серверные обработчики позволяют выполнять определенные действия по расписанию.
\hypertarget{Handlers.Schedule.CronExpression.Description}{anchor
  text}Расписание выставляется с помощью cron-выражения вида * * * * *. Это
числа
либо * (звёздочка), которая означает любое число. Порядок:
\begin{itemize}
  \item Минута
  \item Час
  \item День месяца
  \item Месяц
  \item День недели
\end{itemize}
Например * * * * * означает, что обработчик будет запускаться каждую минуту
каждого часа каждого дня каждого месяца каждого дня недели, а
* * 1 * * означает, что обработчик будет запускаться каждую минуту каждого часа
первого числа каждого месяца каждого дня недели
TODO: В разработке
\begin{itemize}
  \item \hyperlink{DeepCase.InsertLink.Description}{Вставьте} связь типа
        \hyperlink{Core.SyncTextFile.Description}{SyncTextFile} с таким
        значением:
        \begin{lstlisting}
(deep) => {
  await deep.insert({
    type_id: await deep.id("@deep-foundation/core", "SyncTextFile"),
    string: {
      data: {
        value: "Hi!"
      }
    }
    in: {
      data: {
        type_id: await deep.id("@deep-foundation/core", "Contain"),
        from_id: await deep.id("deep", "admin"),
      }
    }
  })
}
    \end{lstlisting}
        и \hyperlink{FAQ.HowToSetName}{назовите} её syncTextFile
  \item \hyperlink{DeepCase.InsertLink.Description}{Вставьте} связь типа
        \hyperlink{Core.Handler.Description}{Handler} от
        \hyperlink{Core.dockerSupportsJs.Description}{dockerSupportsJs}
        до syncTextFile и \hyperlink{FAQ.HowToSetName}{назовите} её
        myHandler
  \item \hyperlink{DeepCase.InsertLink.Description}{Вставьте} связь
        типа \hyperlink{Core.Schedule.Description}{Schedule} и
        \hyperlink{FAQ.HowToSetName}{назовите} её mySchedule
  \item \hyperlink{DeepCase.UpdateLink.Description}{Обновите}
        значение связи mySchedule на
        \hyperlink{Handler.Schedule.CronExpression.Description}{* * * * *,
          являющееся
          cron-выражением}
        В нашем случае мы используем * * * * *, что означает, что мы хотим
        запускать обработчик каждую минуту каждого часа каждого дня каждого
        месяца
        каждого дня недели
  \item \hyperlink{DeepCase.InsertLink.Description}{Вставьте} связь
        типа \hyperlink{Core.HandleSchedule.Description}{Schedule} с
        началом mySchedule
        и концом myHandler	и \hyperlink{FAQ.HowToSetName}{назовите} её
        myHandleSchedule
  \item Теперь каждую минуту у вас будет создаваться связь типа
        \hyperlink{Core.SyncTextFile.Description}{SyncTextFile} с текстом
        "Hi!",
        содержащаяся в пользователе admin
\end{itemize}
