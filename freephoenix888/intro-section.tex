\section{Вступление}
\subsection{Аннотация}

Данный текст представляет собой описание основных концепций и структур в рамках
архитектуры Deep. Deep представляет собой систему, организованную вокруг
понятий связей, пакетов и обработчиков.

\begin{enumerate}
      \item \textbf{Связи:}
            \begin{itemize}
                  \item Связи могут иметь различные атрибуты, такие как
                        идентификатор,
                        тип, начало, конец и значение.
                  \item Тип, начало и конец связи являются связями, что создает
                        структуру, где связи могут указывать друг на друга.
                  \item Приведены примеры различных типов связей, таких как
                        Message,
                        Conversation, Reply и др.
            \end{itemize}

      \item \textbf{Связи-типы:}
            \begin{itemize}
                  \item Любая связь может использоваться как тип для создания
                        другой
                        связи
                  \item Если связь является точкой (без начала и конца) -
                        экземпляр
                        такой связи тоже будет точкой
                  \item Если связь ведёт от связи А к связи Б, то экземпляр
                        этой связи
                        должен вести от экземпляра связи А к экземпляру связи Б
                  \item Если нужно что бы экземпляр связи мог быть связан с
                        любой
                        связью,
                        то нужно использовать встроенный тип
                        \hyperlink{Core.Any.Description}{Any}
            \end{itemize}

      \item \textbf{Пакеты:}
            \begin{itemize}
                  \item Существует встроенный в Deep тип
                        \hyperlink{Core.Contain.Description}{Contain},
                        обозначающий
                        принадлежность связей к определенному пакету.
                  \item Пакеты не обязательно содержат связи напрямую; они
                        могут
                        указывать на связи из других пакетов. В таком случае
                        пакет
                        становится зависимым от другого пакета
            \end{itemize}

      \item \textbf{Обработчики:}
            \begin{itemize}
                  \item Виды серверных обработчиков: обработчики базы данных,
                        обработчики веб-маршрута, обработчики порта
                  \item Серверные обработчики баз данных и событий над связями
                        могут
                        быть синхронными (выполняются в рамках транзакции и
                        внутри базы
                        данных) и
                        асинхронными (выполняются в Docker'е, за пределами
                        транзакции
                        внутри базы
                        данных)
                  \item Клиентские обработчики предоставляют возможность
                        показывать
                        визуальный графический интерфейс для связи в
                        \hyperlink{DeepCase.Description}{DeepCase}
                  \item Серверные обработчики по расписанию позволяют выполнять
                        определенные задачи по расписанию
            \end{itemize}

      \item \textbf{\hyperlink{DeepCase.Description}{DeepCase:}}
            \begin{itemize}
                  \item \hyperlink{DeepCase.Description}{DeepCase} это
                        графический
                        визуальный
                        интерфейс Deep
                  \item Визуализация конкретных связей реализована при помощи
                        клиентских обработчиков
                  \item \hyperlink{DeepCase.Description}{DeepCase} позволяет
                        взаимодействовать
                        со связями: вставлять,
                        обновлять, удалять, просматривать, делать выборку
                        связей,
                        путешествовать по ним, используя traveller, и ещё
                        множество
                        возможностей
                  \item \hyperlink{DeepCase.Description}{DeepCase} это npm
                        пакет,
                        который
                        позволяет встраивать его
                        компоненты, включая отдельные клиентские обработчики,
                        используя
                        ClientHandler
                        компонент в любой проект, использующий npm пакет React
            \end{itemize}

      \item \textbf{Дип клиент:}
            \begin{itemize}
                  \item Дип Клиент для программного взаимодействия с Deep в
                        среде
                        JavaScript, позволяет вставлять, получать, обновлять и
                        удалять
                        связи.
            \end{itemize}

      \item \textbf{Таблицы:}
            \begin{itemize}
                  \item Deep использует использует различные таблицы, такие как
                        links
                        (для хранения связей) и numbers, strings, objects (для
                        хранения
                        значений
                        привязанных к связям)
            \end{itemize}
\end{enumerate}

Приведены примеры использования Дип Клиента для различных операций, таких как
вставка связей различных типов, получение данных, обновление и удаление.

% Add a subsection heading to the table of contents
% \addtocontents{toc}{\protect\subsection{Содержание}}

% Table of contents
\tableofcontents