\subsection{Материализованный путь, деревья}
\subsubsection{Описание}
Для каждой связи выстраивается материализованный путь. Деревья используют эту
концепцию и позволяют упрощать взаимодействие с группами связей.
Самым простым и понятным примером является
\hyperlink{containTree.Deletion.Example}{Удаление ниже по дереву containTree}
Вы сами можете создавать деревья!
\subsubsection{Примеры}
\paragraph{Создание дерева}
\begin{itemize}
      \item \hyperlink{DeepCase.InsertLink.Description}{Вставьте} связь типа
            \hyperlink{Tree}{Tree} и \hyperlink{FAQ.HowToSetName}{назовите} её
            myTree
      \item \hyperlink{DeepCase.InsertLink.Description}{Вставьте} связь типа
            \hyperlink{Type.Description}{Type} и
            \hyperlink{FAQ.HowToSetName}{назовите} её MyNode1
      \item \hyperlink{DeepCase.InsertLink.Description}{Вставьте} связь типа
            \hyperlink{Type.Description}{Type} и
            \hyperlink{FAQ.HowToSetName}{назовите} её MyNode2
      \item \hyperlink{DeepCase.InsertLink.Description}{Вставьте} связь типа
            \hyperlink{Type.Description}{Type}, с началом
            \hyperlink{Core.Any.Description}{Any}, концом
            \hyperlink{Core.Any.Description}{Any} и
            \hyperlink{FAQ.HowToSetName}{назовите} её
            MyDownLink
      \item \hyperlink{DeepCase.InsertLink.Description}{Вставьте} связь типа
            \hyperlink{Type.Description}{Type}, с началом
            \hyperlink{Core.Any.Description}{Any}, концом
            \hyperlink{Core.Any.Description}{Any} и
            \hyperlink{FAQ.HowToSetName}{назовите} её
            MyUpLink
      \item \hyperlink{DeepCase.InsertLink.Description}{Вставьте} связь
            \hyperlink{Core.TreeIncludeNode.Description}{TreeIncludeNode} от
            экземпляра \hyperlink{Core.Tree.Description}{Tree} до MyNode1
      \item \hyperlink{DeepCase.InsertLink.Description}{Вставьте} связь
            \hyperlink{Core.TreeIncludeNode.Description}{TreeIncludeNode} от
            экземпляра \hyperlink{Core.Tree.Description}{Tree} до MyNode2
      \item \hyperlink{DeepCase.InsertLink.Description}{Вставьте} связь
            \hyperlink{Core.TreeIncludeDown.Description}{TreeIncludeDown} от
            экземпляра \hyperlink{Core.Tree.Description}{Tree} до MyDownLink
      \item \hyperlink{DeepCase.InsertLink.Description}{Вставьте} связь
            \hyperlink{Core.TreeIncludeNode.Description}{TreeIncludeUp} от
            экземпляра \hyperlink{Core.Tree.Description}{Tree} до MyUpLink
      \item \hyperlink{DeepCase.InsertLink.Description}{Вставьте} связь типа
            MyNode1
            и \hyperlink{FAQ.HowToSetName}{назовите} её myNode1
      \item \hyperlink{DeepCase.InsertLink.Description}{Вставьте} связь типа
            MyNode2
            и \hyperlink{FAQ.HowToSetName}{назовите} её myNode2
      \item \hyperlink{DeepCase.InsertLink.Description}{Вставьте} связь типа
            MyDownLink с началом myNode1 и концом myNode2 и
            \hyperlink{FAQ.HowToSetName}{назовите} её myDownLink
      \item \hyperlink{DeepCase.InsertLink.Description}{Вставьте} связь типа
            MyUpLink с началом myNode2 и концом myNode1 и
            \hyperlink{FAQ.HowToSetName}{назовите} её myUpLink
      \item \hyperlink{DeepCase.InsertLink.Description}{Вставьте} связь типа
            \hyperlink{Core.Space.Description}{Space}, что бы пеерстать видеть
            наши
            связи и показать
            работу деревьев, используя Traveller
      \item Открываем контекстное меню экземпляра
            \hyperlink{Core.Space.Description}{Space} и
            нажимаем кнопку "Space" что бы перейти в это пространство
      \item Открываем контекстное меню, нажимаем кнопку Query, вводим в поле
            для
            ввода myNode1, нажимаем на myNode1, нажимаем галочку, что бы
            увидеть
            эту связь
      \item Открываем контекстное меню, нажимаем кнопку
            \hyperlink{Traveller.Description}{Traveller}, нажимаем кнопку Down,
            нажимаем на
            кнопку
            myTree
      \item Созерцайте путешествие от связи myNode1 вниз по дереву myTree, вы
            должны видеть связи myNode1, myDownLink, myNode2
      \item Открываем меню DeepCase, нажимаем крестик около слова Space, что бы
            выйти из текущего пространства и вернуться в пространство нашего
            пользователя
      \item Снова создаём новое пространство, заходим в него
      \item Снова с помощью Query запрашиваем связь, но уже myNode2
      \item С помощью \hyperlink{Traveller.Description}{Traveller} запрашиваем
            связи вверх
            по дереву myTree
      \item Видим связи myNode2, myDownLink, myNode1
      \item Возвращаемся в предыдущее пространство
\end{itemize}
