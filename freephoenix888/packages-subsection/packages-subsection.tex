\subsection{Пакеты}
\subsubsection{Описание}
Пакет предназначены для упаковки связей вместе в одну связь, которую можно
выложить, скачать, точно так-же, как вы устанавливаете какое либо программное
обеспечение состоящее из различных файлов. \\
Пакет это тоже связь, экземпляр \hyperlink{Core.Package.Description}{Package} \\
Принадлежность связи к пакету определяется с помощью экземпляра
\hyperlink{Core.Contain.Description}{Contain}. То есть что бы ваша связь
находилась в  
пакетее - нужно иметь экземпляр \hyperlink{Core.Contain.Description}{Contain},
ведущий
от пакета к связи \\
Если связь пакета А указывает началом/концом/типом на связь пакета Б, то пакет
А становится зависимым от пакета Б \\
\subsubsection{Примеры}
\subparagraph{Создание пакета}
\begin{itemize}
      \item \hyperlink{DeepCase.InsertLink.Description}{Вставьте} связь типа
            \hyperlink{Core.Package.Description}{Type}
            \hyperlink{FAQ.HowToInsertLinkWithoutFromAndTo}{без начала и конца}
            и
            \hyperlink{FAQ.HowToSetName}{назовите} её MyPackage
\end{itemize}
\subparagraph{Добавление связи в пакет}
\begin{itemize}
      \item \hyperlink{DeepCase.InsertLink.Description}{Вставьте} связь типа
            \hyperlink{Core.Package.Description}{Type}
            \hyperlink{FAQ.HowToInsertLinkWithoutFromAndTo}{без начала и конца}
            и
            \hyperlink{FAQ.HowToSetName}{назовите} её MyPackage. Кстати, теперь
            ваш
            пакет
            зависит от пакета
            \hyperlink{Core.Description}{@deep-foundation/core},
            потому что
            использует его связь \hyperlink{Core.Type.Description}{Type} как
            тип
            для нашей связи
      \item \hyperlink{DeepCase.InsertLink.Description}{Вставьте} связь типа
            \hyperlink{Core.Type.Description}{Type}
            \hyperlink{FAQ.HowToInsertLinkWithoutFromAndTo}{без
                  начала и конца} и \hyperlink{FAQ.HowToSetName}{назовите} её
            MyNode1
\end{itemize}
\subparagraph{Публикация пакета}
\begin{itemize}
      \item Что бы опубликовать пакет нужно
            \hyperlink{DeepCase.InsertLink.Description}{вставить} связь типа
            \hyperlink{NpmPackager.Token.Description}{Type}
            \hyperlink{FAQ.HowToInsertLinkWithoutFromAndTo}{без начала и конца}
            и
            дать ей
            значение равное вашему токену из npm, который вы можете получить на
            \url{https://www.npmjs.com/}
      \item Нажмите левой кнопкой мыши на MyPackage и увидите клиентский
            обработчик, показывающий визуализацию для пакета. Введите название
            и
            версию
            пакета и нажмите кнопку Save, а затем Publish
      \item Теперь нам нужно увидеть связь Publish, которая создалась благодаря
            нажатию на кнопку Publish. Откройте контекстное меню MyPackage,
            нажмите
            Traveller->Out->Publish
      \item \hyperlink{Handlers.Async.HowToGetResult}{Просмотрите результат
                  асинхронного хендлера для связи Publish}
\end{itemize}

% Обработчики
