\subsection{Таблицы}
Deep использует различные таблицы, такие как links (для хранения связей) и
numbers, strings, objects (для хранения значений привязанных к связям).

\begin{itemize}
      \item \textbf{links} - связи
      \item \textbf{strings} - строковые значения
      \item \textbf{numbers} - числовые значения
      \item \textbf{objects} - объектные значения (фактически JSON, поэтому
            можно
            вставлять числа/строки в таблицу objects, а не только объекты)
      \item \textbf{selectors} - селекторы
      \item \textbf{mp} - материализация пути связей
      \item \textbf{tree} - деревья
      \item \textbf{\texttt{promise\_links}} - для очереди выполнения
            обработчиков
      \item \textbf{\texttt{handlers}} - обработчики
      \item \textbf{\texttt{reserved}} - зарезервированные связи
      \item \textbf{\texttt{bool\_exp}} - используется для компиляции
            дополнительных условий в селекторах
      \item \textbf{\texttt{can}} - позволяет ответить быстро на вопрос есть ли
            у
            той или иной связи прав
      \item \textbf{\texttt{files}} - файлы
\end{itemize}


